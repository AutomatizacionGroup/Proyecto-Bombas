\documentclass[12pt]{article}
\usepackage[utf8]{inputenc}
\usepackage[spanish]{babel}
\usepackage[margin = 1.5in]{geometry}
\usepackage{graphicx}
\graphicspath{{/home/smichelena/Desktop/Automatizacion/proyecto_bombas_andres/}}

\title{Implementación de SCADA en sistema de bombas domesticó}
\author{Grupo Automatizacion}

\begin{document}
	
	\maketitle
	
	\section{Objetivo}
	
	El objeto de este proyecto inicial es familiarizarse con las distintas tecnologías, tanto de software como de hardware, que la compañía utilizara para implementar soluciones de IoT a nivel industrial.
	
	\section{Tecnologías}
	
	\subsection{Hardware}
	Se utilizara un PLC de NodeMCU, el cual cuenta con un modulo de WiFi, 10 pines de GPIO y un firmware hecho para implementar soluciones con IoT.
	
	\begin{figure}[h]
		\centering
		\caption{pinout del PLC de NodeMCU}
		\includegraphics[width = 5cm, height = 5cm]{plc_nodemcu}
	\end{figure}
	
	\subsection{software}
	En cuanto a software, se utilizara Ignition como SCADA, el cual tiene un modulo que implementa MQTT como protocolo de comunicación entre el PLC y el servidor de SCADA.
	
	
	\begin{figure}[h]
		\centering
		\caption{Arquitectura estándar de Ignition}
		\includegraphics[width=10cm, height=8cm]{arc_igtion}
	\end{figure}
	
	\section{Objetivos}
	
	\begin{enumerate}
		\item Lograr comunicación bidireccional entre el PLC de NodeMCU y el servidor de Ignition
		\item Hacer análisis de data obtenida usando la plataforma de Ignition
		\item Crear algoritmo que ajuste parámetros de funcionamiento automáticamente
		\item Implementar sistema de alarmas en base a funcionamiento del sistema (ej, Esta entrando agua, etc)
	\end{enumerate}
	
	\section{arquitectura a usar}
	
	
\end{document}